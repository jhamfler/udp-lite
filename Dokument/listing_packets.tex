%--------------------------------------------------------------------
% Pakete
%--------------------------------------------------------------------

% KOMA-Script
\usepackage[table]{xcolor}          % Erweitertes Farbpaket mit vielen Farbmodellen (notwenidig zur Farbdefinition)
\usepackage{listings}               % Quellcode einbinden und formatieren
\renewcommand{\lstlistlistingname}{Verzeichnis der Listings}
\definecolor{darkgreen}{rgb}{0,.4,0}
\definecolor{darkviolett}{rgb}{.4,0,.4}
\newcommand{\listingswidth}{\textwidth-1cm}
\lstset{
    language=C,                  % oder  C++, Pascal, {[77]Fortran}, ...
    numbers=left,                   % Position der Zeilennummerierung
    firstnumber=auto,               % Erste  Zeilennummer
    basicstyle=\ttfamily\small,     % Textgröße  des Standardtexts
    keywordstyle=\ttfamily\color{darkviolett},    % Formattierung Schlüsselwörter
    commentstyle=\ttfamily\color{darkgreen},       % Formattierung Kommentar
    stringstyle=\ttfamily\color{blue},             % Formattierung Strings
    %keywordstyle=\ttfamily\color{black},
    %commentstyle=\ttfamily\color{black},
    %stringstyle=\ttfamily\color{black},
    numberstyle=\tiny,              % Textgröße der Zeilennummern
    stepnumber=1,                   % Angezeigte Zeilennummern
    numbersep=5pt,                  % Abstand zw. Zeilennummern und Code
    aboveskip=15pt,                 % Abstand oberhalb des Codes
    belowskip=11pt,                 % Abstand unterhalb des Codes
    captionpos=b,                   % Position der Überschrift
    linewidth=15.8cm,               % TODO \textwidth-2em
    xleftmargin=10pt,               % Linke Einrückung
    frame=single,                   % Rahmentyp
    breaklines=true,                % Umbruch langer Zeilen
    showstringspaces=false          % Spezielles Zeichen für Leerzeichen
}