%Verwendete Codierung
\usepackage[utf8]{inputenc}

%Definierung der Seitenränder, , Kopf- und Fußzeile
\usepackage[left=2.5cm,right=2.5cm,top=2cm,bottom=2.2cm]{geometry}

%Zur Einbindung von Grafiken
\usepackage{graphicx}

% Dieses Paket bietet LaTeX die Möglichkeit Text um ein Bild anzuordnen.
\usepackage{wrapfig}

%Zum einfügen von Sonderzeichn und Währungen mittels \textxxxx befhels
\usepackage{textcomp}

% Paket erzeugt ein anklickbares Verzeichnis in der PDF-Datei.
\usepackage{hyperref}

% Paket für 1,5-zeiligen Zeilenabstand.
% \usepackage{setspace}

% Mit dem Multicol-Paket können Textteile nebeneinander arrangiert werden.
\usepackage{multicol}

% Das Acronym-Paket wird für das Abkürzungsverzeichnis benötigt.
\usepackage{acronym}

%Ermöglicht die automatische Trennung von Worten mit Umlauten
\usepackage[T1]{fontenc}

 % um u. a. Zeilenumbrüche bei zu langen URLs im Literaturverzeichnis zu erzwingen
%\usepackage[colorlinks=false, pdfborder={0 0 0}]{hyperref}

%Einbindung der Sprachpakete zur Rechtschreibkorrektur und Silbentrennung
\usepackage[english, ngerman]{babel}