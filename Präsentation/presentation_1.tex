%%%%%%%%%%%%%%%%%%%%%%%%%%%%%%%%%%%%%%%%%
% Beamer Presentation
% LaTeX Template
% Version 1.0 (10/11/12)
%
% This template has been downloaded from:
% http://www.LaTeXTemplates.com
%
% License:
% CC BY-NC-SA 3.0 (http://creativecommons.org/licenses/by-nc-sa/3.0/)
%
%%%%%%%%%%%%%%%%%%%%%%%%%%%%%%%%%%%%%%%%%

%----------------------------------------------------------------------------------------
%	PACKAGES AND THEMES
%----------------------------------------------------------------------------------------

\documentclass{beamer}

\mode<presentation> {

% The Beamer class comes with a number of default slide themes
% which change the colors and layouts of slides. Below this is a list
% of all the themes, uncomment each in turn to see what they look like.

%\usetheme{default}
% % %\usetheme{AnnArbor}
%\usetheme{Antibes}
%\usetheme{Bergen}
%\usetheme{Berkeley}
% %\usetheme{Berlin}
%\usetheme{Boadilla}
% %\usetheme{CambridgeUS}
% %\usetheme{Copenhagen}
\usetheme{Darmstadt}
%\usetheme{Dresden}
%\usetheme{Frankfurt}
%\usetheme{Goettingen}
%\usetheme{Hannover}
%\usetheme{Ilmenau}
%\usetheme{JuanLesPins}
%\usetheme{Luebeck}
%\usetheme{Madrid}
%\usetheme{Malmoe}
%\usetheme{Marburg}
%\usetheme{Montpellier}
%\usetheme{PaloAlto}
%\usetheme{Pittsburgh}
%\usetheme{Rochester}
%\usetheme{Singapore}
%\usetheme{Szeged}
%\usetheme{Warsaw}

% As well as themes, the Beamer class has a number of color themes
% for any slide theme. Uncomment each of these in turn to see how it
% changes the colors of your current slide theme.

%\usecolortheme{albatross}
%\usecolortheme{beaver}
%\usecolortheme{beetle}
%\usecolortheme{crane}
%%%\usecolortheme{dolphin}
%\usecolortheme{dove}
%\usecolortheme{fly}
%\usecolortheme{lily}
%\usecolortheme{orchid}
%\usecolortheme{rose}
%\usecolortheme{seagull}
%%%\usecolortheme{seahorse}
% %\usecolortheme{whale}
% %%\usecolortheme{wolverine}

%\setbeamertemplate{footline} % To remove the footer line in all slides uncomment this line
%\setbeamertemplate{footline}[page number] % To replace the footer line in all slides with a simple slide count uncomment this line

%\setbeamertemplate{navigation symbols}{} % To remove the navigation symbols from the bottom of all slides uncomment this line
}
\usepackage{verbatim} % für Mehrzeilen-Kommentare
\usepackage[english, ngerman]{babel} % deutsche sprache
\usepackage{graphicx} % Allows including images
\usepackage{booktabs} % Allows the use of \toprule, \midrule and \bottomrule in tables
\usepackage[utf8x]{inputenc}
\usepackage{listings}               % Quellcode einbinden und formatieren
\renewcommand{\lstlistlistingname}{Verzeichnis der Listings}
\definecolor{darkgreen}{rgb}{0,.4,0}
\definecolor{darkviolett}{rgb}{.4,0,.4}
\newcommand{\listingswidth}{\textwidth-1cm}
\lstset{
    language=C,                  % oder  C++, Pascal, {[77]Fortran}, ...
    numbers=left,                   % Position der Zeilennummerierung
    firstnumber=auto,               % Erste  Zeilennummer
    basicstyle=\ttfamily\small,     % Textgröße  des Standardtexts
    keywordstyle=\ttfamily\color{darkviolett},    % Formattierung Schlüsselwörter
    commentstyle=\ttfamily\color{darkgreen},       % Formattierung Kommentar
    stringstyle=\ttfamily\color{blue},             % Formattierung Strings
    %keywordstyle=\ttfamily\color{black},
    %commentstyle=\ttfamily\color{black},
    %stringstyle=\ttfamily\color{black},
    numberstyle=\tiny,              % Textgröße der Zeilennummern
    stepnumber=1,                   % Angezeigte Zeilennummern
    numbersep=5pt,                  % Abstand zw. Zeilennummern und Code
    aboveskip=15pt,                 % Abstand oberhalb des Codes
    belowskip=11pt,                 % Abstand unterhalb des Codes
    captionpos=b,                   % Position der Überschrift
    linewidth=15.8cm,               % TODO \textwidth-2em
    xleftmargin=10pt,               % Linke Einrückung
    frame=single,                   % Rahmentyp
    breaklines=true,                % Umbruch langer Zeilen
    showstringspaces=false          % Spezielles Zeichen für Leerzeichen
}

%----------------------------------------------------------------------------------------
%	TITLE PAGE
%----------------------------------------------------------------------------------------

\title[UDP-Lite]{Lightweight User Datagram Protocol} % The short title appears at the bottom of every slide, the full title is only on the title page

\author{Johannes Hamfler} % Your name
\institute[HfTL] % Your institution as it will appear on the bottom of every slide, may be shorthand to save space
{
Hochschule für Telekommunikation Leipzig \\ % Your institution for the title page
\medskip
\textit{johannes.hamfler@hftl.de} % Your email address
}
\date{\today} % Date, can be changed to a custom date

\begin{document}

\begin{frame}
\titlepage % Print the title page as the first slide
\end{frame}

\begin{frame}
\frametitle{Übersicht} % Table of contents slide, comment this block out to remove it
\tableofcontents % Throughout your presentation, if you choose to use \section{} and \subsection{} commands, these will automatically be printed on this slide as an overview of your presentation
\end{frame}

%----------------------------------------------------------------------------------------
%	PRESENTATION SLIDES
%----------------------------------------------------------------------------------------
%------------------------------------------------
%------------------------------------------------

\section{Einleitung}
\begin{frame}
%\frametitle{Einleitung}
\begin{itemize}
\item UDP: RFC 768
	\begin{itemize}
	\item verbindungsloses Protokoll
	\item Sprachdienste, Videokommunikation und Echtzeitübertragung
	\item verlorene und fehlerhafte Datenpakete werden nicht erneut übertragen
	\item Prüfsumme nicht gesetzt oder über gesamtes Paket
	\end{itemize}
\end{itemize}
\par
\begin{itemize}
\item UDP-Lite: RFC 3828
\item Option
	\begin{itemize}
	\item fehlerhafte Pakete an höhere Schichten weiterleiten
	\end{itemize}
\item Vorteil
	\begin{itemize}
	\item Codec verarbeitet korrekte Bits
	\item nützlich für Anwendungen
	\end{itemize}
\end{itemize}
\par
\begin{itemize}
\item OSI-Schicht: 4 - Transportschicht
\item für Ausnutzung der Stärken sind höhere Schichten notwendig
\end{itemize}
\end{frame}

% % % % % % % % % % % % % % % % % % % % % % % % % % % % % % % % % % % % % % % % % % % % % % % % % % % % % % % % % % % % % % %

\section{Protokollspezifikation}
\subsection{Protokollformat}
\begin{frame}
\begin{itemize}
\item Einteilung des Payloads
	\begin{itemize}
	\item ein Teil mit Fehlerkorrekturwert
	\item ein Teil ohne Prüfsumme möglich
	\end{itemize}
\item Prüfsummenteil
	\begin{itemize}
	\item üblicherweise Steuerinformationen
	\item wenn Beschädigt wird das Paket in Schicht 4 verworfen
	\end{itemize}
\item Teil ohne Prüfsumme
	\begin{itemize}
	\item unkritische Informationen wie z.B. Sprachdaten
	\item wenn Fehler vorhanden sind wird das Paket nicht verworfen
	\end{itemize}
\item Prüfsumme ist über das gesamte Paket möglich
	\begin{itemize}
	\item semantisch gleich zu UDP
	\end{itemize}
\end{itemize}
\end{frame}


\begin{frame}
\begin{itemize}
\item UDP-Lite kann Verbesserungen beim Decoder für folgende Codecs bieten
	\begin{itemize}
	\item AMR speech codec [RFC-3267]
	\item Internet Low Bit Rate Codec [ILBRC]
	\item error resilient H.263+ [ITU-H.263]
	\item H.264 [ITU-H.264; H.264]
	\item MPEG-4 [ISO-14496] video codecs)
	\end{itemize}
\item dynamische Umschaltung zwischen UDP- und UDP-Lite-Verhalten gewünscht
\item Verbindungen sollten sich ihrer Fehleranfälligkeit bewusst sein
\end{itemize}
\end{frame}


\begin{frame}[fragile]
\begin{itemize}
\item UDP-Lite-Header
\end{itemize}
\begin{lstlisting}[linewidth=0.90\textwidth=small]
     0              15 16             31
    +--------+--------+--------+--------+
    |     Source      |   Destination   |
    |      Port       |      Port       |
    +--------+--------+--------+--------+
    |    Checksum     |                 |
    |    Coverage     |    Checksum     |
    +--------+--------+--------+--------+
    |                                   |
    :              Payload              :
    |                                   |
    +-----------------------------------+
\end{lstlisting}
\begin{itemize}
\item Unterschied zu UDP
	\begin{itemize}
	\item Length-Feld wurde zum Cecksum-Coverage-Feld
		\begin{itemize}
		\item gibt Länge an, bis wohin die Prüfsumme berechnet wird
		\end{itemize}
	\item Länge des Pakets wird aus IP-Paket entnommen
	\end{itemize}
\end{itemize}
\end{frame}

% % % % % % % % % % % % % % % % % % % % % % % % % % % % % % % % % % % % % % % % % % % % % % % % % % % % % % % % % % % % % % %

\subsection{Checksum-Coverage-Feld und Prüfsumme}
\begin{frame}
\begin{itemize}
\item 8 Bit = 1 Byte = 1 Inkrement des Checksum-Coverage-Felds
\item Anfang des Zählens beim 1. Oktett der PDU
\item zugelassene Werte
	\begin{itemize}
	\item 0 -- Prüfsumme über das gesamte Paket anwenden
	\item $ 8 $ bis maximale Länge eines IP-Pakets -- teilweise Prüfsumme möglich
	\end{itemize}
\item nicht zugelassene Werte
	\begin{itemize}
	\item 1 bis 7 -- Paket wird beim Empfänger verworfen
	\end{itemize}
\item Prüfsummenberechnung
	\begin{enumerate}
	\item Einerkomplement der zu prüfenden Informationen aus dem IP-Header
	\item die Summe daraus
	\item 16-Bit-Komplement
	\end{enumerate}
	\item Prüfsummenlänge
		\begin{itemize}
		\item ein Vielfaches von 2 Byte
		\item Paket ist notfalls mit Nullen aufzufüllen
		\end{itemize}
\end{itemize}
\end{frame}

% % % % % % % % % % % % % % % % % % % % % % % % % % % % % % % % % % % % % % % % % % % % % % % % % % % % % % % % % % % % % % %

\subsection{Anwendungsschnittstelle}
\begin{frame}
zu höheren Schichten:
\begin{itemize}
\item gleiche Funktionen wie bei UDP
\item Standardfall
	\begin{itemize}
	\item UDP-Prüfsummenlänge imitieren
	\end{itemize}
\item Zusatzfunktion 
	\begin{itemize}
	\item Prüfsummenlänge an UDP-Lite zu übertragen
	\item über System-Calls Fehlertoleranz mitteilen
	\end{itemize}
\item Anwendung kann durch Codecs besser Fehler beheben
\end{itemize}

zu niedrigeren Schichten:
\begin{itemize}
\item dürfen Pakete nicht verwerfen, außer bei Fehlern im kritischen Teil
\item Checksum-Coverage-Feld sollte ausgelesen werden
\end{itemize}
\begin{itemize}
\item Internet Protokoll
	\begin{itemize}
	\item Länge des IP-Payloads zur Größenberechnung
	\item IP-Paket darf nicht mit Padding-Bytes aufgefüllt werden
	\end{itemize}
\end{itemize}
\end{frame}

% % % % % % % % % % % % % % % % % % % % % % % % % % % % % % % % % % % % % % % % % % % % % % % % % % % % % % % % % % % % % % %

\subsection{Sicherheitsbetrachtungen}
\begin{frame}
\begin{itemize}
\item IPv6 verlangt eine Fehlerkorrektur für UDP
\end{itemize}
\begin{itemize}
\item IPSec mit ESP bringt mit UDP-Lite keinen Vorteil
	\begin{itemize}
	\item keine Erkennung des Payloads
	\end{itemize}
\end{itemize}
\begin{itemize}
\item Lösung
	\begin{itemize}
	\item Verschlüsselung auf Transportschicht
	\item Stromchiffren anstatt Blockchiffren
		\begin{itemize}
		\item Fehlerspreizung wird vermindert
		\item vorhersagbare Manipulationen am Payload möglich
		\end{itemize}
	\end{itemize}
\end{itemize}
\end{frame}


% % % % % % % % % % % % % % % % % % % % % % % % % % % % % % % % % % % % % % % % % % % % % % % % % % % % % % % % % % % % % % %

\section{Zusammenfassung}

\begin{frame}

\begin{itemize}
\item nur geringe Änderungen an Anwendungen nötig
\end{itemize}

\begin{itemize}
\item zur Nutzung wie UDP muss Prüfsumme über gesamtes Paket angewandt werden
\end{itemize}

\begin{itemize}
\item IP-Protokoll-ID 136 zur Erkennung des Payloads nutzen
\end{itemize}

\end{frame}



\begin{comment}
\begin{frame}
\frametitle{Blocks of Highlighted Text}
\begin{block}{Block 1}
Lorem ipsum dolor sit amet, consectetur adipiscing elit. Integer lectus nisl, ultricies in feugiat rutrum, porttitor sit amet augue. Aliquam ut tortor mauris. Sed volutpat ante purus, quis accumsan dolor.
\end{block}

\begin{block}{Block 2}
Pellentesque sed tellus purus. Class aptent taciti sociosqu ad litora torquent per conubia nostra, per inceptos himenaeos. Vestibulum quis magna at risus dictum tempor eu vitae velit.
\end{block}

\begin{block}{Block 3}
Suspendisse tincidunt sagittis gravida. Curabitur condimentum, enim sed venenatis rutrum, ipsum neque consectetur orci, sed blandit justo nisi ac lacus.
\end{block}
\end{frame}

%------------------------------------------------


\begin{frame}
\frametitle{Multiple Columns}
\begin{columns}[c] % The "c" option specifies centered vertical alignment while the "t" option is used for top vertical alignment

\column{.45\textwidth} % Left column and width
\textbf{Heading}
\begin{enumerate}
\item Statement
\item Explanation
\item Example
\end{enumerate}

\column{.5\textwidth} % Right column and width
Lorem ipsum dolor sit amet, consectetur adipiscing elit. Integer lectus nisl, ultricies in feugiat rutrum, porttitor sit amet augue. Aliquam ut tortor mauris. Sed volutpat ante purus, quis accumsan dolor.

\end{columns}
\end{frame}

%------------------------------------------------
%\section{Second Section}
%------------------------------------------------

\begin{frame}
\frametitle{Table}
\begin{table}
\begin{tabular}{l l l}
\toprule
\textbf{Treatments} & \textbf{Response 1} & \textbf{Response 2}\\
\midrule
Treatment 1 & 0.0003262 & 0.562 \\
Treatment 2 & 0.0015681 & 0.910 \\
Treatment 3 & 0.0009271 & 0.296 \\
\bottomrule
\end{tabular}
\caption{Table caption}
\end{table}
\end{frame}

%------------------------------------------------

\begin{frame}
\frametitle{Theorem}
\begin{theorem}[Mass--energy equivalence]
$E = mc^2$
\end{theorem}
\end{frame}

%------------------------------------------------

\begin{frame}[fragile] % Need to use the fragile option when verbatim is used in the slide
\frametitle{Verbatim}
\begin{example}[Theorem Slide Code]
\begin{verbatim}
\begin{frame}
\frametitle{Theorem}
\begin{theorem}[Mass--energy equivalence]
$E = mc^2$
\end{theorem}
\end{frame}\end{verbatim}
\end{example}
\end{frame}

%------------------------------------------------

\begin{frame}
\frametitle{Figure}
Uncomment the code on this slide to include your own image from the same directory as the template .TeX file.
%\begin{figure}
%\includegraphics[width=0.8\linewidth]{test}
%\end{figure}
\end{frame}

%------------------------------------------------

\begin{frame}[fragile] % Need to use the fragile option when verbatim is used in the slide
\frametitle{Citation}
An example of the \verb|\cite| command to cite within the presentation:\\~

This statement requires citation \cite{p1}.
\end{frame}

%------------------------------------------------
\end{comment}


\begin{frame}
\frametitle{Referenzen}
\footnotesize{
\begin{thebibliography}{99} % Beamer does not support BibTeX so references must be inserted manually as below
\bibitem[RFC]{p1} RFC 3828 (Juli 2004), 
\newblock The Lightweight User Datagram Protocol (UDP-Lite)
\newblock \emph{http://tools.ietf.org/html/rfc3828}
\end{thebibliography}
}
\end{frame}

%------------------------------------------------

\begin{frame}
\Huge{\centerline{Vielen Dank}}
\Huge{\centerline{für eure Aufmerksamkeit.}}
\Huge{\centerline{}}
\Huge{\centerline{Sind Fragen offen?}}
\end{frame}

%----------------------------------------------------------------------------------------

\end{document}