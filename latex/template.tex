% This is "sig-alternate.tex" V2.0 May 2012
% This file should be compiled with V2.5 of "sig-alternate.cls" May 2012
%
% This example file demonstrates the use of the 'sig-alternate.cls'
% V2.5 LaTeX2e document class file. It is for those submitting
% articles to ACM Conference Proceedings WHO DO NOT WISH TO
% STRICTLY ADHERE TO THE SIGS (PUBS-BOARD-ENDORSED) STYLE.
% The 'sig-alternate.cls' file will produce a similar-looking,
% albeit, 'tighter' paper resulting in, invariably, fewer pages.
%
% ----------------------------------------------------------------------------------------------------------------
% This .tex file (and associated .cls V2.5) produces:
%       1) The Permission Statement
%       2) The Conference (location) Info information
%       3) The Copyright Line with ACM data
%       4) NO page numbers
%
% as against the acm_proc_article-sp.cls file which
% DOES NOT produce 1) thru' 3) above.
%
% Using 'sig-alternate.cls' you have control, however, from within
% the source .tex file, over both the CopyrightYear
% (defaulted to 200X) and the ACM Copyright Data
% (defaulted to X-XXXXX-XX-X/XX/XX).
% e.g.
% \CopyrightYear{2007} will cause 2007 to appear in the copyright line.
% \crdata{0-12345-67-8/90/12} will cause 0-12345-67-8/90/12 to appear in the copyright line.
%
% ---------------------------------------------------------------------------------------------------------------
% This .tex source is an example which *does* use
% the .bib file (from which the .bbl file % is produced).
% REMEMBER HOWEVER: After having produced the .bbl file,
% and prior to final submission, you *NEED* to 'insert'
% your .bbl file into your source .tex file so as to provide
% ONE 'self-contained' source file.
%
% ================= IF YOU HAVE QUESTIONS =======================
% Questions regarding the SIGS styles, SIGS policies and
% procedures, Conferences etc. should be sent to
% Adrienne Griscti (griscti@acm.org)
%
% Technical questions _only_ to
% Gerald Murray (murray@hq.acm.org)
% ===============================================================
%
% For tracking purposes - this is V2.0 - May 2012

\documentclass{template}
\usepackage[utf8x]{inputenc}
%--------------------------------------------------------------------
% Pakete
%--------------------------------------------------------------------

% KOMA-Script
\usepackage[table]{xcolor}          % Erweitertes Farbpaket mit vielen Farbmodellen (notwenidig zur Farbdefinition)
\usepackage{listings}               % Quellcode einbinden und formatieren
\renewcommand{\lstlistlistingname}{Verzeichnis der Listings}
\definecolor{darkgreen}{rgb}{0,.4,0}
\definecolor{darkviolett}{rgb}{.4,0,.4}
\newcommand{\listingswidth}{\textwidth-1cm}
\lstset{
    language=C,                  % oder  C++, Pascal, {[77]Fortran}, ...
    numbers=left,                   % Position der Zeilennummerierung
    firstnumber=auto,               % Erste  Zeilennummer
    basicstyle=\ttfamily\small,     % Textgröße  des Standardtexts
    keywordstyle=\ttfamily\color{darkviolett},    % Formattierung Schlüsselwörter
    commentstyle=\ttfamily\color{darkgreen},       % Formattierung Kommentar
    stringstyle=\ttfamily\color{blue},             % Formattierung Strings
    %keywordstyle=\ttfamily\color{black},
    %commentstyle=\ttfamily\color{black},
    %stringstyle=\ttfamily\color{black},
    numberstyle=\tiny,              % Textgröße der Zeilennummern
    stepnumber=1,                   % Angezeigte Zeilennummern
    numbersep=5pt,                  % Abstand zw. Zeilennummern und Code
    aboveskip=15pt,                 % Abstand oberhalb des Codes
    belowskip=11pt,                 % Abstand unterhalb des Codes
    captionpos=b,                   % Position der Überschrift
    linewidth=15.8cm,               % TODO \textwidth-2em
    xleftmargin=10pt,               % Linke Einrückung
    frame=single,                   % Rahmentyp
    breaklines=true,                % Umbruch langer Zeilen
    showstringspaces=false          % Spezielles Zeichen für Leerzeichen
}

\begin{document}
%
% --- Author Metadata here ---
\conferenceinfo{Hochschule für Telekommunikation Leipzig\\Modul:Protokolle\\WS 2014/15}{}
%\CopyrightYear{2007} % Allows default copyright year (20XX) to be over-ridden - IF NEED BE.
%\crdata{0-12345-67-8/90/01}  % Allows default copyright data (0-89791-88-6/97/05) to be over-ridden - IF NEED BE.
% --- End of Author Metadata ---

\title{UDP-Lite {\ttlit}}
%
% You need the command \numberofauthors to handle the 'placement
% and alignment' of the authors beneath the title.
%
% For aesthetic reasons, we recommend 'three authors at a time'
% i.e. three 'name/affiliation blocks' be placed beneath the title.
%
% NOTE: You are NOT restricted in how many 'rows' of
% "name/affiliations" may appear. We just ask that you restrict
% the number of 'columns' to three.
%
% Because of the available 'opening page real-estate'
% we ask you to refrain from putting more than six authors
% (two rows with three columns) beneath the article title.
% More than six makes the first-page appear very cluttered indeed.
%
% Use the \alignauthor commands to handle the names
% and affiliations for an 'aesthetic maximum' of six authors.
% Add names, affiliations, addresses for
% the seventh etc. author(s) as the argument for the
% \additionalauthors command.
% These 'additional authors' will be output/set for you
% without further effort on your part as the last section in
% the body of your article BEFORE References or any Appendices.

\numberofauthors{1} %  in this sample file, there are a *total*
% of EIGHT authors. SIX appear on the 'first-page' (for formatting
% reasons) and the remaining two appear in the \additionalauthors section.
%
\author{
% You can go ahead and credit any number of authors here,
% e.g. one 'row of three' or two rows (consisting of one row of three
% and a second row of one, two or three).
%
% The command \alignauthor (no curly braces needed) should
% precede each author name, affiliation/snail-mail address and
% e-mail address. Additionally, tag each line of
% affiliation/address with \affaddr, and tag the
% e-mail address with \email.
%
% 1st. author
\alignauthor
Johannes Hamfler\\
       \affaddr{KMI12}\\
       \email{johannes.hamfler@hft-leipzig.de}
% 2nd. author
%\alignauthor
%2nd author
%       \affaddr{affiliation}\\
%       \email{e-mail}
% 3rd. author
% \alignauthor Lars Th{\o}rv{\"a}ld\titlenote{This author is the
% one who did all the really hard work.}\\
%        \affaddr{The Th{\o}rv{\"a}ld Group}\\
%        \affaddr{1 Th{\o}rv{\"a}ld Circle}\\
%        \affaddr{Hekla, Iceland}\\
%        \email{larst@affiliation.org}
% \and  % use '\and' if you need 'another row' of author names
% % 4th. author
% \alignauthor Lawrence P. Leipuner\\
%        \affaddr{Brookhaven Laboratories}\\
%        \affaddr{Brookhaven National Lab}\\
%        \affaddr{P.O. Box 5000}\\
%        \email{lleipuner@researchlabs.org}
% % 5th. author
% \alignauthor Sean Fogarty\\
%        \affaddr{NASA Ames Research Center}\\
%        \affaddr{Moffett Field}\\
%        \affaddr{California 94035}\\
%        \email{fogartys@amesres.org}
% % 6th. author
% \alignauthor Charles Palmer\\
%        \affaddr{Palmer Research Laboratories}\\
%        \affaddr{8600 Datapoint Drive}\\
%        \affaddr{San Antonio, Texas 78229}\\
%        \email{cpalmer@prl.com}
}
% There's nothing stopping you putting the seventh, eighth, etc.
% author on the opening page (as the 'third row') but we ask,
% for aesthetic reasons that you place these 'additional authors'
% in the \additional authors block, viz.
%\additionalauthors{Additional authors: John Smith (The Th{\o}rv{\"a}ld Group,
%email: {\texttt{jsmith@affiliation.org}}) and Julius P.~Kumquat
%(The Kumquat Consortium, email: {\texttt{jpkumquat@consortium.net}}).}
%\date{13 Dezember 2014}
% Just remember to make sure that the TOTAL number of authors
% is the number that will appear on the first page PLUS the
% number that will appear in the \additionalauthors section.

\maketitle
\begin{abstract}
In diesem Dokument wird das Leightweight User Datagram Protokoll (UDP-Lite) beschrieben, welches ähnlich UDP ist. Der Focus dieses Dokuments liegt in der Beschreibung der Vorteile, die UDP-Lite gegenüber UDP aufweisen kann. Des Weiteren wird das Protokoll in das ISO OSI-Referenzmodell eingeordnet und die Auswirkungen auf andere Schichten in diesem beschrieben.

------------------------------
 \LaTeX\ {\em alternate} \textit{bla}
\end{abstract}

% A category with the (minimum) three required fields
\category{H.4}{Information Systems Applications}{Miscellaneous}
%A category including the fourth, optional field follows...
\category{D.2.8}{Software Engineering}{Metrics}[complexity measures, performance measures]

\terms{Theory}

\keywords{UDP-Lite}

\section{Einleitung}

UDP, welches im RFC 768 beschrieben ist wurde jahrelang als verbindungsloses Protokoll verwendet und ist weit verbreitet. Ach heute hat dieses große Bedeutung für Sprachdienste, Videokommunikation und Echtzeitübertragung. Der Vorteil des Protokolls gegenüber TCP liegt vor allem bei der Sprachkommunikation darin, dass verlorene und fehlerhafte Datenpakete nicht erneut übertragen werden, da diese nach wenigen Millisekunden schon nicht mehr von Bedeutung sind. UDP-Lite versucht das Problem der fehlerhaften Pakete, welche beim Empfänger gelöscht werden, zu mindern, indem die Option besteht fehlerhafte Pakete dennoch zu verwenden und an höhere Schichten weiterleiten zu können. Bei Sprachdiensten hätte dies den Vorteil, dass der in einer höheren Schicht angesiedelte Codec die korrekten Bits auf eine bestimmte weise verarbeitet, so dass diese nützlich für die Anwendung sind. Fehlerhafte Bits könnten für den Codec ebenfalls einen Nutzen darstellen, so dass UDP-Lite in diesem Zusammenhang einen Vorteil darstellen würde.



-----------------------------------------
\textit{proceedings}
(18 $\times$ 23.5 cm [7" $\times$ 9.25"])
\footnote{Two of these, the {\texttt{\char'134 numberofauthors}}}
{\texttt{\char'134 alignauthor}}
{\texttt{\char'134 balancecolumns}}

%#############################################

In RFC 3828 findet sich ..

Da manche Codecs die Fähigkeit besitzen 
beschädigten Payload zu behandeln und nützliche Informationen
aus diesem zu extrahieren, wurde bei UDP-Lite ein Paket in
zwei Teile aufgegliedert werden. Ein Teil kann mit einem Fehlerkorrekturwert
überprüft werden, um die Integrität der darin enthaltenen Daten zu sichern,
ein anderer Teil kann ohne Prüfsumme vorhanden sein.

In dem Teil, in welchem
eine Fehlerüberprüfung stattfinden soll, werden üblicherweise Steuerinformationen
übertragen, welche unbedingt fehlerfrei vorhanden sein müssen, 
um die Parameter des Payloads beim Empfänger richtig interpretieren zu können.
Sollte der Payload in diesem Teil beschädigt sein, wird das Paket beim Empfänger in der
Transportschicht verworfen.

Der andere Teil des Payloads, welcher beim klassischen UDP üblicherweise Daten enthält,
welche nicht zwingend neu übertragen werden müssen, kann ohne Fehlerkorrektur übertragen werden,
damit die darüber liegenden Schichten auch beschädigte Daten bearbeiten können
um aus diesen ebenfalls nützliche Informationen für eine Anwendung zu extrahieren.
Da in diesem Teil nicht überprüft wird ob Fehler vorhanden sind,
wird der Payload nicht für die Entscheidung der Weiterleitung an höhere Schichten verwendet.

Wird eine Prüfsumme über das gesamte Paket angewandt, so ist UDP-Lite semantisch identisch
zu UDP.

Im RFC wurden Beobachtungen erläutert, welche hier kurz erwähnt werden.

Es wurden folgende Codecs als Beispiele genannt, welche mit UDP-Lite
eine Verbesserung der decodierten Daten erreichen können:
AMR speech codec [RFC-3267]
Internet Low Bit Rate Codec [ILBRC]
error resilient H.263+ [ITU-H.263]
H.264 [ITU-H.264; H.264]
MPEG-4 [ISO-14496] video codecs)

Des Weiteren ist es nützlich, wenn niedrigere Schichten beschädigte IP Pakete
weiterleiten, wenn dies verlangt wird. Sollten Verbindungen sich ihrer
Fehleranfälligkeit bewusst sein, so ist es möglich, dass eine physische Verbindung
eine höhere Sicherheit für sensible Daten gewährleisten, was durch verschiedene
Fehlerkorrekturverfahren erreicht werden kann.

Außerdem sollte die Transport- und Vermittlungsschicht höher gelegene
Applikationen nicht an ihrer Ausführung hindern, weil Pakete beschädigt sind.
UDP eigent sich deshalb nur bedingt, da bei diesem die Prüfsumme gesetzt
sein muss. Bei IP ist dies nicht der Fall.



\section{Protokollübersicht {\secit UDP-Lite}}

Nachfolgend ist der UDP-Header abgebildet.

\begin{lstlisting}[linewidth=0.47\textwidth]
     0              15 16             31
    +--------+--------+--------+--------+
    |     Source      |   Destination   |
    |      Port       |      Port       |
    +--------+--------+--------+--------+
    |    Checksum     |                 |
    |    Coverage     |    Checksum     |
    +--------+--------+--------+--------+
    |                                   |
    :              Payload              :
    |                                   |
    +-----------------------------------+
\end{lstlisting}
\cite{rfc:udplite}

Dieser unterscheidet sich von dem UDP-Header in der Hinsicht, dass
das Length-Feld mit einem Cecksum-Coverage-Feld ausgestattet wurde.
Dieses ist dazu da, um die Länge anzugeben, bis wohin die Prüfsumme
berechnet wird. Dies war möglich, da die Information über die Länge
des Pakets aus IP-Paketen entnommen werden kann.

\subsection{Beschreibung der Felder der PDU}

Das Source- und Destination-Port-Feld sind dem von UDP gleich, wobei
das Checksum-Coverage-Feld die Länge in Oketetten angibt, die von
der Prüfsumme einbezogen werden. Hierbei wird ab dem ersten Oktett
mit dem Zählen angefangen. 

Der Header muss dabei immer mit einer Prüfsumme gesichert werden.
Eine Prüfsumme von 0 bedeutet dabei, dass das gesamte Paket in die
Prüfsumme einbezogen wird. Die Prüfsumme des UDP-Header muss 0 oder
mindestens 8 sein, was bedeutet, dass die 8 Bytes des Headers beinhaltet
sein müssen. Ein Paket mit einer Prüfsummenlänge zwischen 1 und 7 muss
beim Empfänger verworfen werden, da dieses die Bedingung nicht erfüllt.
Das berechnete Prüfsummenfeld muss einen Preuso-Header enthalten welcher
auf IP basiert. UDP-Lite Pakete die eine gößere Prüfsummenlänge als
der IP-Header annehmen kann müssen ebenfalls verworfen werden.

Da das Prüfsummenfeld ein 16-Bit-Komplement-Einerkomplement der Summe
des Einerkomplements des Pseudo-Headers darstellt, muss das UDP-Lite-Paket
im Payload ein vielfaches von 2Byte aufweisen. Notfalls wird dieses mit
Nullen aufgefüllt. Die Informationen, aus welchen die Prüfsumme berechnet
wird, werden dem IP-header entnommen. Bevor die Prüfsumme berechnet wird
muss das Prüfsumenfeld jedoch auf 0 gesetzt werden. Sollte nach der
Berechnung die Prüfsumme 0 ergeben, so werden 16 Einsen übertragen.

Da manche Anwendungen die UDP-Lite benutzen möglicherweise keine
Fehlerbehandlung wünschen, kann hier die Prüfsummenlänge einfach
auf 8 gesetzt werden um den Payload nicht mit einbeziehen zu müssen.
Dadurch wird sichergestellt, dass die Steuerinformationen in jedem
Fall korrekt ankommen.


\subsection{Der Pseudo-Header}

Der Pseudo-Header von UDP-Lite unterscheidet sich von dem in UDP
insofern, dass der Wert des Längenfeldes nicht von UDP-Lite-Header genommen wird,
sondern von Informationen aus den IP-Paketen. Dabei wird die
Berechnung gleich wie bei TCP ausgeführt, was bedeutet, dass
nicht nur der Payload, sondern auch der Header von UDP-Lite mit
einbezogen wird. Dadurch, dass die Länge auf diese Weise berechnet,
muss das Prüfsummenfeld bei 8 beginnen und ermöglicht so einen
einfachen softwareseitigen Vergleich.


\subsection{Die Anwendungsschnittstelle}

Die Anwendungsschnittstelle stellt die gleichen Funktionen wie
bei UDP zur Verfügung. Des Weiteren sollte eine Möglichkeit der
sendenden Anwendung bestehen, den Prüfsummenlängenwert an UDP-Lite
zu übertragen. Zumindest sollte jedoch die Möglichkeit für eine
empfangende Anwendung bestehen, die Weiterleitung von Paketen mit
Prüfsummenlängen kleiner als eine Festgelegte zu blockieren.

Im RFC wird empfohlen, dass UDP-Lite standardmäßig das Verhalten
von UDP imitieren sollte, indem das Prüfsummenlängenfeld der
Länge des UDP-Lite-Pakets entsprechen soll, um das gesamte Paket
verifizieren zu können. Über einen expliziten Aufruf, einem
sogenannten System Call, beim Sender sollen Anwendungen die 
fehlertolerant sind UDP-Lite ihre Fehlertoleranz
mitteilen. Eine empfangende Anwendung die ebenfalls eine teilweise
angewandte Prüfsumme nutzen wollen, sollte dies ebenfalls über
einen einen solchen Aufruf kund geben.

Da im Internet Pfade variieren und verschiedene Eigenschaften aufweisen,
können keine pauschalen Aussagen über die Fehlerschemas einer
Verbindung gemacht werden. Deshalb sollten Anwendungen die UDP-Lite
nutzen keine Annahmen der Fehler in einem UDP-Lite-Paket machen,
solange der Bereich nicht in die Berechnung der Prüfsumme
mit einbezogen wurde. Anwendungen sollten deshalb, wenn nötig
ihre eigenen Fehlerprüfmechnismen nutzen.


\subsection{Die IP-Schnittstelle}

Wie bei UDP muss auch UDP-Lite den Pseudo-Header der 
IP-Implementierung erhalten, welcher die IP-Adresse 
und Protokollfelder des IP-Headers beinhaltet, sowie die
Länge des IP-Payloads welche aus dem Längenfeld der IP-Pakete
entnommen werden kann.
Der Sender darf dabei nicht den IP-Payload mit Padding-Bytes
auffüllen, da die Länge des UDP-Lite-Pakets daraus entnommen werden soll.


\subsection{IP-Jumbo-PDUs}

Da das Prüfsummenlängenfeld Werte bis 65535 annehmen kann,
können genaue Prüfsummenlängen benutzt werden. Dies ist bei
Jumbo-PDUs (Jumbogrammen) nicht der Fall. Es kann entweder das
gesamte Paket mit der Prüfsummenlänge von 0 oder alle Oktette bis
zum 65535ten Oktett einschließen.


\subsection{Betrachtung der niedrigeren Schichten}

Frames die UDP-Lite-Pakete enthalten dürfen von niedrigeren
Schichten nicht verworfen werden, da die Fehlerbehandlung in einer
höheren Schicht erfolgt. Eine Ausnahme wäre der Fall, wenn ein
Fehler im sensiblen Datenbereich existiert. Das Prüfsummenlängenfeld
könnte für Verbindungen, welche partielle Fehlererkennung ermöglichen dafür benutzt werden,
niedrigeren Schichten mitzuteilen,
in welchen Bereichen Fehler auftreten dürfen. Da der sensible
Teil des UDP-Lite-Pakets zwischen dem ersten Oktet des IP-Headers
und dem letzten Oktet liegt, welcher vom Prüfsummenlängenfeld
bekannt gegeben wird, kann der sensible Teil in der gleichen Weise
behandelt werden, wie ein UDP-Paket.

Da Verbindungen, welche keine partielle Fehlererkennung ermöglichen,
in einem Fehlerfall das Paket verwerfen müssen, wird das UDP-Lite-Paket
in gleicher Weise wie ein UDP-Paket behandelt.

Somit lässt sich bei UDP-Lite sagen, dass dieses Protokoll nur eine
Verbesserung erwirken kann, wenn in Schicht 2 des OSI-Referenzmodells
die Partielle Prüfsumme und die Prüfsummenlänge von UDP-Lite genutzt wird.
Dies würde seine Wirkung jedoch erst in Fehleranfälligen
Umgebungen entfalten.









-------------------------------------------------------------
\texttt{{\char'134}section}
\footnote{This is the second footnote.  It
starts a series of three footnotes that add nothing
informational, but just give an idea of how footnotes work
and look. It is a wordy one, just so you see
how a longish one plays out.}
\textbf{document} 


\subsection{Type Changes and {\subsecit Special} Characters}
We have already seen several typeface changes in this sample.  You
can indicate italicized words or phrases in your text with
the command \texttt{{\char'134}textit}; emboldening with the
command \texttt{{\char'134}textbf}
and typewriter-style (for instance, for computer code) with
\texttt{{\char'134}texttt}.  But remember, you do not
have to indicate typestyle changes when such changes are
part of the \textit{structural} elements of your
article; for instance, the heading of this subsection will
be in a sans serif\footnote{A third footnote, here.
Let's make this a rather short one to
see how it looks.} typeface, but that is handled by the
document class file. Take care with the use
of\footnote{A fourth, and last, footnote.}
the curly braces in typeface changes; they mark
the beginning and end of
the text that is to be in the different typeface.

You can use whatever symbols, accented characters, or
non-English characters you need anywhere in your document;
you can find a complete list of what is
available in the \textit{\LaTeX\
User's Guide}\cite{Lamport:LaTeX}.

\subsection{Math Equations}
You may want to display math equations in three distinct styles:
inline, numbered or non-numbered display.  Each of
the three are discussed in the next sections.

\subsubsection{Inline (In-text) Equations}
A formula that appears in the running text is called an
inline or in-text formula.  It is produced by the
\textbf{math} environment, which can be
invoked with the usual \texttt{{\char'134}begin. . .{\char'134}end}
construction or with the short form \texttt{\$. . .\$}. You
can use any of the symbols and structures,
from $\alpha$ to $\omega$, available in
\LaTeX\cite{Lamport:LaTeX}; this section will simply show a
few examples of in-text equations in context. Notice how
this equation: \begin{math}\lim_{n\rightarrow \infty}x=0\end{math},
set here in in-line math style, looks slightly different when
set in display style.  (See next section).

\subsubsection{Display Equations}
A numbered display equation -- one set off by vertical space
from the text and centered horizontally -- is produced
by the \textbf{equation} environment. An unnumbered display
equation is produced by the \textbf{displaymath} environment.

Again, in either environment, you can use any of the symbols
and structures available in \LaTeX; this section will just
give a couple of examples of display equations in context.
First, consider the equation, shown as an inline equation above:
\begin{equation}\lim_{n\rightarrow \infty}x=0\end{equation}
Notice how it is formatted somewhat differently in
the \textbf{displaymath}
environment.  Now, we'll enter an unnumbered equation:
\begin{displaymath}\sum_{i=0}^{\infty} x + 1\end{displaymath}
and follow it with another numbered equation:
\begin{equation}\sum_{i=0}^{\infty}x_i=\int_{0}^{\pi+2} f\end{equation}
just to demonstrate \LaTeX's able handling of numbering.

\subsection{Citations}
Citations to articles \cite{bowman:reasoning,
clark:pct, braams:babel, herlihy:methodology},
conference proceedings \cite{clark:pct} or
books \cite{salas:calculus, Lamport:LaTeX} listed
in the Bibliography section of your
article will occur throughout the text of your article.
You should use BibTeX to automatically produce this bibliography;
you simply need to insert one of several citation commands with
a key of the item cited in the proper location in
the \texttt{.tex} file \cite{Lamport:LaTeX}.
The key is a short reference you invent to uniquely
identify each work; in this sample document, the key is
the first author's surname and a
word from the title.  This identifying key is included
with each item in the \texttt{.bib} file for your article.

The details of the construction of the \texttt{.bib} file
are beyond the scope of this sample document, but more
information can be found in the \textit{Author's Guide},
and exhaustive details in the \textit{\LaTeX\ User's
Guide}\cite{Lamport:LaTeX}.

This article shows only the plainest form
of the citation command, using \texttt{{\char'134}cite}.
This is what is stipulated in the SIGS style specifications.
No other citation format is endorsed or supported.

\subsection{Tables}
Because tables cannot be split across pages, the best
placement for them is typically the top of the page
nearest their initial cite.  To
ensure this proper ``floating'' placement of tables, use the
environment \textbf{table} to enclose the table's contents and
the table caption.  The contents of the table itself must go
in the \textbf{tabular} environment, to
be aligned properly in rows and columns, with the desired
horizontal and vertical rules.  Again, detailed instructions
on \textbf{tabular} material
is found in the \textit{\LaTeX\ User's Guide}.

Immediately following this sentence is the point at which
Table 1 is included in the input file; compare the
placement of the table here with the table in the printed
dvi output of this document.

\begin{table}
\centering
\caption{Frequency of Special Characters}
\begin{tabular}{|c|c|l|} \hline
Non-English or Math&Frequency&Comments\\ \hline
\O & 1 in 1,000& For Swedish names\\ \hline
$\pi$ & 1 in 5& Common in math\\ \hline
\$ & 4 in 5 & Used in business\\ \hline
$\Psi^2_1$ & 1 in 40,000& Unexplained usage\\
\hline\end{tabular}
\end{table}

To set a wider table, which takes up the whole width of
the page's live area, use the environment
\textbf{table*} to enclose the table's contents and
the table caption.  As with a single-column table, this wide
table will ``float" to a location deemed more desirable.
Immediately following this sentence is the point at which
Table 2 is included in the input file; again, it is
instructive to compare the placement of the
table here with the table in the printed dvi
output of this document.


\begin{table*}
\centering
\caption{Some Typical Commands}
\begin{tabular}{|c|c|l|} \hline
Command&A Number&Comments\\ \hline
\texttt{{\char'134}alignauthor} & 100& Author alignment\\ \hline
\texttt{{\char'134}numberofauthors}& 200& Author enumeration\\ \hline
\texttt{{\char'134}table}& 300 & For tables\\ \hline
\texttt{{\char'134}table*}& 400& For wider tables\\ \hline\end{tabular}
\end{table*}
% end the environment with {table*}, NOTE not {table}!

\subsection{Figures}
Like tables, figures cannot be split across pages; the
best placement for them
is typically the top or the bottom of the page nearest
their initial cite.  To ensure this proper ``floating'' placement
of figures, use the environment
\textbf{figure} to enclose the figure and its caption.

This sample document contains examples of \textbf{.eps}
and \textbf{.ps} files to be displayable with \LaTeX.  More
details on each of these is found in the \textit{Author's Guide}.

\begin{figure}
\centering
\epsfig{file=fly.eps}
\caption{A sample black and white graphic (.eps format).}
\end{figure}

\begin{figure}
\centering
\epsfig{file=fly.eps, height=1in, width=1in}
\caption{A sample black and white graphic (.eps format)
that has been resized with the \texttt{epsfig} command.}
\end{figure}


As was the case with tables, you may want a figure
that spans two columns.  To do this, and still to
ensure proper ``floating'' placement of tables, use the environment
\textbf{figure*} to enclose the figure and its caption.
and don't forget to end the environment with
{figure*}, not {figure}!

\begin{figure*}
\centering
\epsfig{file=flies.eps}
\caption{A sample black and white graphic (.eps format)
that needs to span two columns of text.}
\end{figure*}

Note that either {\textbf{.ps}} or {\textbf{.eps}} formats are
used; use
the \texttt{{\char'134}epsfig} or \texttt{{\char'134}psfig}
commands as appropriate for the different file types.

\begin{figure}
\centering
%\psfig{file=rosette.ps, height=1in, width=1in,}
\caption{A sample black and white graphic (.ps format) that has
been resized with the \texttt{psfig} command.}
\vskip -6pt
\end{figure}

\subsection{Theorem-like Constructs}
Other common constructs that may occur in your article are
the forms for logical constructs like theorems, axioms,
corollaries and proofs.  There are
two forms, one produced by the
command \texttt{{\char'134}newtheorem} and the
other by the command \texttt{{\char'134}newdef}; perhaps
the clearest and easiest way to distinguish them is
to compare the two in the output of this sample document:

This uses the \textbf{theorem} environment, created by
the\linebreak\texttt{{\char'134}newtheorem} command:
\newtheorem{theorem}{Theorem}
\begin{theorem}
Let $f$ be continuous on $[a,b]$.  If $G$ is
an antiderivative for $f$ on $[a,b]$, then
\begin{displaymath}\int^b_af(t)dt = G(b) - G(a).\end{displaymath}
\end{theorem}

The other uses the \textbf{definition} environment, created
by the \texttt{{\char'134}newdef} command:
\newdef{definition}{Definition}
\begin{definition}
If $z$ is irrational, then by $e^z$ we mean the
unique number which has
logarithm $z$: \begin{displaymath}{\log e^z = z}\end{displaymath}
\end{definition}

Two lists of constructs that use one of these
forms is given in the
\textit{Author's  Guidelines}.
 
There is one other similar construct environment, which is
already set up
for you; i.e. you must \textit{not} use
a \texttt{{\char'134}newdef} command to
create it: the \textbf{proof} environment.  Here
is a example of its use:
\begin{proof}
Suppose on the contrary there exists a real number $L$ such that
\begin{displaymath}
\lim_{x\rightarrow\infty} \frac{f(x)}{g(x)} = L.
\end{displaymath}
Then
\begin{displaymath}
l=\lim_{x\rightarrow c} f(x)
= \lim_{x\rightarrow c}
\left[ g{x} \cdot \frac{f(x)}{g(x)} \right ]
= \lim_{x\rightarrow c} g(x) \cdot \lim_{x\rightarrow c}
\frac{f(x)}{g(x)} = 0\cdot L = 0,
\end{displaymath}
which contradicts our assumption that $l\neq 0$.
\end{proof}

Complete rules about using these environments and using the
two different creation commands are in the
\textit{Author's Guide}; please consult it for more
detailed instructions.  If you need to use another construct,
not listed therein, which you want to have the same
formatting as the Theorem
or the Definition\cite{salas:calculus} shown above,
use the \texttt{{\char'134}newtheorem} or the
\texttt{{\char'134}newdef} command,
respectively, to create it.

\subsection*{A {\secit Caveat} for the \TeX\ Expert}
Because you have just been given permission to
use the \texttt{{\char'134}newdef} command to create a
new form, you might think you can
use \TeX's \texttt{{\char'134}def} to create a
new command: \textit{Please refrain from doing this!}
Remember that your \LaTeX\ source code is primarily intended
to create camera-ready copy, but may be converted
to other forms -- e.g. HTML. If you inadvertently omit
some or all of the \texttt{{\char'134}def}s recompilation will
be, to say the least, problematic.

\section{Conclusions}
This paragraph will end the body of this sample document.
Remember that you might still have Acknowledgments or
Appendices; brief samples of these
follow.  There is still the Bibliography to deal with; and
we will make a disclaimer about that here: with the exception
of the reference to the \LaTeX\ book, the citations in
this paper are to articles which have nothing to
do with the present subject and are used as
examples only.
%\end{document}  % This is where a 'short' article might terminate

%ACKNOWLEDGMENTS are optional
\section{Acknowledgments}
This section is optional; it is a location for you
to acknowledge grants, funding, editing assistance and
what have you.  In the present case, for example, the
authors would like to thank Gerald Murray of ACM for
his help in codifying this \textit{Author's Guide}
and the \textbf{.cls} and \textbf{.tex} files that it describes.

%
% The following two commands are all you need in the
% initial runs of your .tex file to
% produce the bibliography for the citations in your paper.
\bibliographystyle{abbrv}
\bibliography{template}  % template.bib is the name of the Bibliography in this case
% You must have a proper ".bib" file
%  and remember to run:
% latex bibtex latex latex
% to resolve all references
%
% ACM needs 'a single self-contained file'!
%
%APPENDICES are optional
%\balancecolumns
\appendix
%Appendix A
\section{Headings in Appendices}
The rules about hierarchical headings discussed above for
the body of the article are different in the appendices.
In the \textbf{appendix} environment, the command
\textbf{section} is used to
indicate the start of each Appendix, with alphabetic order
designation (i.e. the first is A, the second B, etc.) and
a title (if you include one).  So, if you need
hierarchical structure
\textit{within} an Appendix, start with \textbf{subsection} as the
highest level. Here is an outline of the body of this
document in Appendix-appropriate form:
\subsection{Introduction}
\subsection{The Body of the Paper}
\subsubsection{Type Changes and  Special Characters}
\subsubsection{Math Equations}
\paragraph{Inline (In-text) Equations}
\paragraph{Display Equations}
\subsubsection{Citations}
\subsubsection{Tables}
\subsubsection{Figures}
\subsubsection{Theorem-like Constructs}
\subsubsection*{A Caveat for the \TeX\ Expert}
\subsection{Conclusions}
\subsection{Acknowledgments}
\subsection{Additional Authors}
This section is inserted by \LaTeX; you do not insert it.
You just add the names and information in the
\texttt{{\char'134}additionalauthors} command at the start
of the document.
\subsection{References}
Generated by bibtex from your ~.bib file.  Run latex,
then bibtex, then latex twice (to resolve references)
to create the ~.bbl file.  Insert that ~.bbl file into
the .tex source file and comment out
the command \texttt{{\char'134}thebibliography}.
% This next section command marks the start of
% Appendix B, and does not continue the present hierarchy
\section{More Help for the Hardy}
The sig-alternate.cls file itself is chock-full of succinct
and helpful comments.  If you consider yourself a moderately
experienced to expert user of \LaTeX, you may find reading
it useful but please remember not to change it.
%\balancecolumns % GM June 2007
% That's all folks!
\end{document}
